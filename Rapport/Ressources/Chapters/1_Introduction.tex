% Chapter Template

\chapter{Introduction} % Main chapter title

\label{Chapitre 1} % Change X to a consecutive number; for referencing this chapter elsewhere, use \ref{ChapterX}

\lhead{ \emph{Introduction}} % Change X to a consecutive number; this is for the header on each page - perhaps a shortened title

Le but de ce travail final était de mettre en pratiques les différents algoritmes de "machine learning" étudiée tout au long des cours que nous avons suivi.\\

Le challenge était de différencier de manière correcte 11 gestes possible. Les données à disposition sont tirées de 4 capteurs posés sur différentes partie du bras (main, poignet, coude et épaule) et d'une caméra kinect (Microsoft). Chacune de ces sources de données nous ont offert un certain nombre de grandeurs physique à exploiter ( accélération, position dans l'espace, quaternions, etc. ).\\

Nous avons choisi deux algorithmes : les réseaux de neurones (ANN) et les chaines de Markov cachées (HMM). Matthieu Favre-Bulle à choisi de travailler sur les HMMs et Michael Mueller à choisi d'utiliser les ANNs. Tout le travail était de trouver les bons paramètres, qui nous permettrons de différencier les différents mouvements possibles.\\

Nous vous souhaitons une bonne lecture de notre travail accompli.
























